\capitulo{4}{Técnicas y herramientas}

%Esta parte de la memoria tiene como objetivo presentar las técnicas metodológicas y las herramientas de desarrollo que se han utilizado para llevar a cabo el proyecto. Si se han estudiado diferentes alternativas de metodologías, herramientas, bibliotecas se puede hacer un resumen de los aspectos más destacados de cada alternativa, incluyendo comparativas entre las distintas opciones y una justificación de las elecciones realizadas. 
%No se pretende que este apartado se convierta en un capítulo de un libro dedicado a cada una de las alternativas, sino comentar los aspectos más destacados de cada opción, con un repaso somero a los fundamentos esenciales y referencias bibliográficas para que el lector pueda ampliar su conocimiento sobre el tema.
En este proyecto he usado:\newline\newline
-Python: Lenguaje de programación en el que se ha desarrollado la aplicación.\newline
-PostgreSQL: Sistema de gestión de bases de datos relacional en el que se almacenan los datos meteorológicos.\newline
-Pandas: Librería de Python que proporciona estructuras de datos y herramientas de análisis de datos.\newline
-NumPy: Librería de Python que proporciona soporte para arrays y matrices multidimensionales.\newline
-Matplotlib: Librería de Python que permite generar gráficos a partir de datos.\newline
-tkinter: Librería de Python que permite crear interfaces gráficas de usuario.\newline
-pysolar: Librería de Python que permite calcular la posición del sol en un determinado lugar y momento.\newline
-sqlchemy: Librería de Python que permite interactuar con bases de datos SQL.\newline
-psycopg2: Librería de Python que permite interactuar con bases de datos PostgreSQL.\newline
Ademas de todo esto he usado Git como sistema de control de versiones y GitHub como plataforma de alojamiento de repositorios remotos.\newline
Tambien he seguido un enfoque de desarrollo ágil, utilizando la metodología Scrum, que me ha permitido adaptar el proyecto a los cambios que han ido surgiendo a lo largo del desarrollo.\newline
\section{Python} 
Python es el lenguaje de programción que he decidido utilizar para desarrollar la aplicación, debido a que es un lenguaje sencillo y fácil de aprender, que cuenta con una gran cantidad de librerías que facilitan el desarrollo de aplicaciones de todo tipo. Además, Python es un lenguaje interpretado, lo que permite ejecutar el código sin necesidad de compilarlo previamente, lo que agiliza el proceso de desarrollo.\newline
Ademas de esto Python es un lenguaje multiplataforma, lo que significa que la aplicación desarrollada en Python se puede ejecutar en cualquier sistema operativo que cuente con un interprete de Python. \newline
Tambien lo he usado por que es un lenguaje muy utilizado en el ámbito de la ciencia de datos, lo que me ha permitido utilizar librerías como Pandas, NumPy y Matplotlib, que me han facilitado el tratamiento y visualización de los datos meteorológicos.\newline
\section{PostgreSQL}
PostgreSQL es un sistema de gestión de bases de datos relacional, que he utilizado para almacenar los datos meteorológicos. He elegido PostgreSQL por que es un sistema de gestión de bases de datos muy potente y fiable, que cuenta con una gran cantidad de funcionalidades que facilitan la gestión de los datos. Además, PostgreSQL es un sistema de gestión de bases de datos de código abierto, lo que significa que es gratuito y se puede modificar y distribuir libremente.\newline
\section{Pandas}
Pandas es una librería de Python que proporciona estructuras de datos y herramientas de análisis de datos. He utilizado Pandas para leer los datos meteorológicos de la base de datos PostgreSQL y para realizar operaciones de limpieza y transformación de los datos. Pandas me ha permitido trabajar con los datos de forma más eficiente y sencilla, ya que proporciona una serie de funciones y métodos que facilitan el tratamiento de los datos.\newline
\section{NumPy}
NumPy es una librería de Python que proporciona soporte para arrays y matrices multidimensionales. He utilizado NumPy para realizar operaciones matemáticas y estadísticas con los datos meteorológicos, ya que NumPy proporciona una serie de funciones y métodos que facilitan la manipulación de arrays y matrices. NumPy me ha permitido realizar cálculos más complejos con los datos de forma más eficiente y sencilla.\newline
\section{Matplotlib}
Matplotlib es una librería de Python que permite generar gráficos a partir de datos. He utilizado Matplotlib para visualizar los datos meteorológicos en forma de gráficos, ya que Matplotlib proporciona una serie de funciones y métodos que facilitan la creación de gráficos. Matplotlib me ha permitido representar los datos de forma más clara y comprensible, lo que ha facilitado la interpretación de los datos.\newline
\section{tkinter}\
tkinter es una librería de Python que permite crear interfaces gráficas de usuario. He utilizado tkinter para desarrollar la interfaz gráfica de la aplicación, ya que tkinter proporciona una serie de widgets y métodos que facilitan la creación de interfaces gráficas. tkinter me ha permitido diseñar una interfaz gráfica sencilla e intuitiva, que ha facilitado la interacción del usuario con la aplicación.\newline
\section{pysolar}
pysolar es una librería de Python que permite calcular la posición del sol en un determinado lugar y momento. He utilizado pysolar para calcular la posición del sol en la central meteorológica de la Universidad de Burgos, ya que pysolar proporciona una serie de funciones y métodos que facilitan el cálculo de la posición del sol. pysolar me ha permitido determinar la altura y el azimut del sol en un determinado momento, lo cual ha sido necesario para poder aplicar los distintos filtros de calidad a los datos meteorológicos.\newline
\section{sqlchemy}
sqlchemy es una librería de Python que permite interactuar con bases de datos SQL. He utilizado sqlchemy debido a su compatibilidad con pandas, ya que atraves de esta libreria he sido capaz de insertar los csv que la central genera y los datos procesados que genera la aplicación de manera sencilla y eficiente.\newline
\section{psycopg2}
psycopg2 es una librería de Python que permite interactuar con bases de datos PostgreSQL. He utilizado psycopg2 para conectarme a la base de datos PostgreSQL y realizar operaciones de lectura y escritura de los datos meteorológicos. psycopg2 me ha permitido interactuar con la base de datos de forma sencilla y eficiente, lo que ha facilitado la gestión de los datos.\newline
\section{Git}
Git es un sistema de control de versiones distribuido, que he utilizado para llevar un control de los cambios realizados en el código fuente de la aplicación. He utilizado Git para crear un repositorio local en el que he ido guardando las distintas versiones del código fuente, y para sincronizar el repositorio local con un repositorio remoto alojado en GitHub. Git me ha permitido llevar un control de los cambios realizados en el código fuente, y revertir los cambios en caso de ser necesario.\newline
\section{GitHub}
GitHub es una plataforma de alojamiento de repositorios remotos, que he utilizado para alojar el repositorio remoto de la aplicación. He utilizado GitHub para sincronizar el repositorio local con el repositorio remoto, lo cual me ha permitido trabajar tanto desde mi portatil como desde mo ordendaor de sobremesa. A su vez GitHub me ha permitido tener una copia de seguridad del código fuente en la nube.\newline
\section{Scrum}
Scrum es una metodología ágil de desarrollo de software, que he utilizado para gestionar el proyecto. He seguido un enfoque de desarrollo ágil, utilizando la metodología Scrum, que me ha permitido adaptar el proyecto a los cambios que han ido surgiendo a lo largo del desarrollo. Scrum me ha permitido dividir el proyecto en iteraciones cortas, llamadas sprints, y priorizar las tareas en función de su importancia y complejidad. Scrum me ha permitido llevar un control de los avances del proyecto, y tomar decisiones en función de los resultados obtenidos en cada sprint.\newline
\section{Conclusiones}
En resumen, he utilizado Python, PostgreSQL, Pandas, NumPy, Matplotlib, tkinter, pysolar, sqlchemy, psycopg2, Git, GitHub y Scrum para desarrollar la aplicación de gestión de datos meteorológicos. Estas herramientas y técnicas me han permitido desarrollar la aplicación de forma eficiente y sencilla, y adaptar el proyecto a los cambios que han ido surgiendo a lo largo del desarrollo. Gracias a estas herramientas y técnicas, he sido capaz de gestionar los datos meteorológicos de la central meteorológica de la Universidad de Burgos de forma más eficiente, sencilla y cómoda, y aprender cómo se trabaja en un proyecto de software real, aplicando los conocimientos adquiridos en la carrera y adquiriendo nuevos conocimientos que me permitirán afrontar problemas reales en el futuro.\newline



